% small.tex
\documentclass[t]{beamer}
\usepackage{graphicx}
\usepackage{listliketab}
\usepackage[absolute,overlay]{textpos}
\usetheme[hideothersubsections]{Goettingen}
\setbeamercolor{palette primary}{bg=structure.fg!25}
\setbeamercolor{palette secondary}{bg=structure.fg!10,fg=black}
\useinnertheme{rectangles} 
\useoutertheme{infolines} 
\author[T.A.E.Smith]{Thomas Smith} 
\title[CM50175]{CM50175\\Research Project Preparation}
\institute[Bath/CS]{Centre for Digital Entertainment\\University of Bath}
\date{April 20, 2014} 
\begin{document}

% Just to remind people that the final presentations should be about your entire project, focussing on the problem, your anticipated solution and the key references from the relevant literature.

%--- the titlepage frame -------------------------%
\begin{frame}
	\titlepage
\end{frame}

%--- the presentation begins here ----------------%
\section{Context}
	\subsection{Born Ready Games}
		\begin{frame}{Born Ready Games}
		\begin{textblock*}{2cm}[0,1](.5cm,9cm) % {block width} (coords)
		\includegraphics[width=2cm]{bornready.jpg}
		\end{textblock*}


		\end{frame}
	\subsection{Strike Suit Zero}
		\begin{frame}{Strike Suit Zero}
		\begin{textblock*}{2cm}[0,1](.5cm,9cm) % {block width} (coords)
		\includegraphics[width=2cm]{strikesuitzero.jpg}
		\end{textblock*}


		\end{frame}
	\subsection{Procedural Destruction}
		\begin{frame}{Procedural Destruction}

		\begin{quote}
		The aim of the project is to explore and implement a declarative approach to the modelling of structure, so that reasoning about the effects of damage can take place over a knowledge-based representation from which a rendering can be synthesized automatically.  The representation evolves over time in response to the damage inflicted, but could also be subject to other forms of failure arising from other environmental events.
		\end{quote}

		\end{frame}
\section{Existing Approaches}
\subsubsection{Art Swap}
\subsubsection{Material-based destruction}
% \subsubsection{Havok Destruction}
% \subsubsection{Euphoria}




\section{Proposed Approach}
\subsection{Answer Set Programming}


% \section{}
% \begin{frame}{Glossary}
% \end{frame}

\end{document}